\section{Komponentenentwurf und Entwicklungsprozess (M. Jud)}
\label{sec:jud}

\subsection{Sie verstehen das Konzept der Software-Komponenten und kennen die Kriterien zur Modularisierung}

\subsubsection{Komponenten}

Eine Software-Komponente ist ein Softwareeinheit, die über Schnittstellen ausgetauscht werden können und damit unabhängig voneinander entwickelt werden können. Weit verbreitete Komponentenmodelle sind z.B. Microsoft .NET oder Enterprise Java Beans.

Eine Komponente kann wiederverwendet oder auch ersetzt werden. Durch die eindeutige Schnittstelle wird die Qualität der Software erhöht. 

In Java sind Komponenten normale Klassen. Diese Klassen können sich an die Java Bean Spezifikation halten. Verteilt werden Komponenten in einer jar-Datei.

\subsubsection{Modularisierung}

Durch Modularisierung wird versucht die Kopplung zu minimieren und die Kohäsion zu maximieren. Modulare Software sollte nach folgenden Kriterien entworfen werden:
\begin{itemize}
	\item \textbf{Zerlegbarkeit} \\
		  Zerlege ein Problem in weniger komplexe Teilprobleme (sollten unabhängig voneinander entwickelbar sein)
	\item \textbf{Kombinierbarkeit} \\
		  Die Software-Elemente sollten sich auch in einem anderen Umfeld einsetzen lassen
	\item \textbf{Verständlichkeit} \\
		  Der Quellcode eines Moduls sollte man verstehen ohne die anderen Module zu kennen
	\item \textbf{Stetigkeit} \\
		  Kleine Änderungen der Anforderungen sollten nur einen kleinen Teil der Module betreffen
\end{itemize}
Folgende Prinzipien sollten ebenfalls beachtet werden:
\begin{itemize}
	\item lose Kopplung (Abbildung \ref{fig:kopplung_kohaesion})
	\item starke Kohäsion (Abbildung \ref{fig:kopplung_kohaesion})
	\item Information Hiding (Geheimnisprinzip)
	\item wenige Schnittstellen
	\item explizite Schnittstellen
\end{itemize}
Man spricht von einer use-Beziehung wenn das korrekte Funktionieren von A von einer korrekten Implementation von B abhängt.

\mygraphics{0.5\textwidth}{fig/kopplung_kohaesion.png}{Zusammenhang zwischen Kopplung und Kohäsion}{kopplung_kohaesion}

\newpage

\subsection{Sie kennen die Kriterien für gute Schnittstellen im Software-Entwurf und können solche Schnittstellen entwerfen}

Schnittstellen vereinfachen die Software-Entwicklung weil:
\begin{itemize}
	\item Schnittstellen machen Software einfacher verständlicher, weil der Benutzer nur die Schnittstelle kennen muss um eine Komponente verwenden zu können.
	\item Schnittstellen helfen auch Abhängigkeiten zu reduzieren.
	\item Schnittstellen erleichtern auch die Wiederverwendbarkeit von bestehender Software.
\end{itemize}
Schnittstellen sollten nach folgenden Kriterien entworfen werden:
\begin{itemize}
	\item Schnittstellen sollen minimal sein (wenig Methoden, wenig Parameter, keine globalen Daten)
	\item Schnittstellen sollen einfach zu verstehen sein
	\item Schnittstellen sollen gut dokumentiert sein
\end{itemize}
Beim \ac{DbC} wird das Zusammenspiel der Komponenten mit einem Vertrag geregelt. Der Vertrag besteht aus:
\begin{itemize}
	\item \textbf{Preconditions} \\
		  Zusicherungen die der Nutzer einzuhalten hat.
	\item \textbf{Postconditions} \\
		  Nachbedingungen die der Anbieter garantiert.
	\item \textbf{Invarianten Bedingung} \\ 
		  Der Anbieter muss sicherstellen, dass Invarianten vor- und nach der Ausführung jeder Methode gelten.
\end{itemize}

\subsection{Sie können verschiedene Arten von Schnittstellen angemessen dokumentieren}

Jede Programmschnittstelle definiert eine Menge von Methoden mit den folgenden Eigenschaften:
\begin{itemize}
	\item \textbf{Syntax} \\
		  Wird mit der verwendeten Programmiersprache beschrieben (Rückgabewerte, Argumente usw.).
	\item \textbf{Semantik} \\
		  Was bewirkt die Methode? Es gibt keine allgemein akzeptierte Form der Dokumentation.
	\item \textbf{Protokoll} \\
		  Wird z.B. ein synchrones oder asynchrones Protokoll verwendet.
	\item \textbf{Nichtfunktionale Eigenschaften} \\
		  Weitere Faktoren wie z.B. Performance, Robustheit, Verfügbarkeit usw.
\end{itemize}
Abbildung \ref{fig:schnittstellen} zeigt die verschiedenen Schnittstellen-Typen.

\mygraphics{0.7\textwidth}{fig/schnittstellen.png}{Schnittstellen-Typen}{schnittstellen}

\newpage

\subsection{Sie können Komponenten entwerfen, dokumentieren, in Java realisieren, testen und deployen}

Abbildung \ref{fig:architekturmuster} zeigt nach welchen Mustern eine komponentenbasierte Architektur entworfen werden kann.

\mygraphics{0.7\textwidth}{fig/architekturmuster.png}{Architekturmuster für Komponenten}{architekturmuster}

Wie Komponenten und Schnittstellen mit UML dokumentiert werden, zeigt Abbildung \ref{fig:komponenten_schnittstellen_uml}.

\mygraphics{0.5\textwidth}{fig/komponenten_schnittstellen_uml.png}{Komponenten und Schnittstellen mit UML}{komponenten_schnittstellen_uml}

Wenn man eine Komponentenarchitektur verwendet muss man das in der Systemspezifikation im Kapital Designentscheide festhalten. Die Kriterien dieser Architektur kommen in das Anforderungsdokument.

\subsection{Sie kennen die Zusammenhänge zwischen Analyse/Design und Test/Abnahme von Softwarekomponenten}

Dieser Zusammenhang lässt sich mit dem V-Modell (Abbildung \ref{fig:v_modell}) besonders gut erklären. Zu jeder Disziplin (Anforderungen, Entwurf, Implementierung) gibt es eine Entsprechung auf der Test-Seite. 

\mygraphics{0.5\textwidth}{fig/v_modell.png}{V-Modell}{v_modell}

Beim Validieren wird sichergestellt dass das richtige Produkt entwickelt wird (z.B. mit Kundenreview). Beim Verifizieren wird überprüft ob das Produkt der Spezifikation entspricht (z.B. mit Integrations-/Systemtests).

\newpage

\subsection{Sie können geeignete Systemtests definieren, diese dokumentieren und die Durchführung protokollieren}

Systemtest prüfen die gesamte Wirkungskette eines Softwareproduktes unter möglichst nahen Einsatzbedingungen. Dadurch werden auch Testfälle abgedeckt die von Unit- und Integrationstests nicht behandelt wurden. Jeder Systemtest muss dokumentiert werden damit bei Änderungen, die alten Tests nochmals durchgeführt werden können (Regression). In der Beschreibung eines Testfalls sollte mindestens folgende Bestandteile enthalten sein:

\begin{itemize}
	\item Vorbedingungen
	\item Testablauf (Was muss ich tun?)
	\item erwartete Ergebnisse (Nachbedingungen)
\end{itemize}

\subsection{Sie wissen welche Informationen über die zu entwickelnde Software wann, wie und wo dokumentiert werden sollen}

\begin{itemize}
	\item Terminplanung gehört in den \ac{PMP}
	\item Design Contraints (Entwurfsvorgaben) gehören in die Kundenanforderung
	\item Erwartete Resultate der Integrationstests gehören in den Testplan
	\item Schnittstellen und UML-Diagramme gehören in die Systemspezifikation
\end{itemize}
Abbildung \ref{fig:doku} zeigt was für wen dokumentiert wird.

\mygraphics{0.5\textwidth}{fig/doku.png}{Dokumentation}{doku}

\newpage

\subsection{Sie kennen Arten, Zweck und Bedeutung von Reviews und können ein Review durchführen und protokollieren}

Bei einem Review werden die Artefakte eines Projekts überprüft. Das Ziel ist Fehler möglichst früh zu erkennen und so Kosten zu sparen. Generell sollen nur immer kleine Teile des Projekts reviewt werden, dann geht die Sitzung auch nicht ewigs. Möglichst früh die High-Level Artefakte reviewen, denn die profitieren am meisten. Tabelle \ref{tab:reviewarten} zeigt die verschiedenen Review-Arten. 

\begin{table}[h!]
	\centering
	\begin{tabular}{|l|c|c|c|c|c|}
		\hline Review Type 		& Planning 	& Preparation 	& Meeting 		& Correction 	& Verification 	\\ 
		\hline Inspection 		& Yes 		& Yes 			& Yes 			& Yes 			& Yes 			\\ 
		\hline Team Review 		& Yes 		& Yes 			& Yes 			& Yes 			& No 			\\ 
		\hline Walkthrough 		& Yes 		& No 			& Yes 			& Yes 			& No 			\\ 
		\hline Pair Programming & Yes 		& No 			& Continuous 	& Yes 			& Yes 			\\ 
		\hline Peer Review 		& No 		& Yes 			& Perhaps 		& Yes 			& No 			\\ 
		\hline Passaround 		& No 		& Yes 			& No 			& Yes 			& No 			\\ 
		\hline AdHoc Review 	& No 		& No 			& Yes 			& Yes 			& No			\\ 
		\hline 
	\end{tabular}
	\caption{Review-Arten}
	\label{tab:reviewarten}
\end{table}

\begin{itemize}
  \item \textbf{Inspection} \\
  Die formalste Art des Reviews. Ist ein strukturierter, wohldefinierter Prozess. Eingesetzt für die kritischsten Architekturen / Entscheide.
  \item \textbf{Team Review} \\
  Eine Sitzung, wo Teile des Projekts genauer unter die Lupe genommen werden, ebenfalls geplant und strukturiert wie die Inspektion, aber nicht so formal.
  \item \textbf{Walkthrough} \\
  Der Autor stellt sein Produkt einer Gruppe von Fachkollegen vor. Ein Szenario und mögliche Alternativen werden durchgespielt.
  \item \textbf{Pair Programming} \\
  Zwei Entwickler arbeiten gemeinsam am selben Produkt.
  \item \textbf{Peer Review} \\
  Ein Arbeitskollege überprüft das Ergebnis
  \item \textbf{Passaround} \\
  Ein Dokument wird vorbereitet und an verschiedene Personen versandt und deren Feedback eingeholt. Ist zeitlich sehr effizient, benötigt aber einiges an Disziplin.
  \item \textbf{Ad Hoc Review} \\
  Normaler Bestandteil der Teamarbeit.
\end{itemize}

\subsection{Sie kennen Bedeutung, Begrifflichkeit und Methoden des Konfigurationsmanagements}\label{sec:konfigurationsmanagement}

In einem System sind alle Teilsysteme (Software, Hardware usw.) Änderungen unterworfen. Weil nicht alle Versionen miteinander funktionieren ist ein Konfigurationsmanagement notwendig. \emph{A \ac{SCI} is an aggregation of software that is designated for configuration management and is treated as a single entity in the \ac{SCM} process. Items that are controlled by \ac{SCM} in addition to the code itself include:}

\begin{itemize}
	\item \emph{plans}
	\item \emph{specifications}
	\item \emph{code libraries}
	\item \emph{data and data dictionaries}
	\item \emph{documentation for installation, maintenance, operations etc.}
\end{itemize} 
Folgende Begriffe sind im \ac{SCM} wichtig:
\begin{itemize}
	\item \textbf{Version} \\
		  Eine Version ist ein spezifisches, identifizierbares Artefakt auf einem bestimmten Entwicklungsstand.
	\item \textbf{Revision} \\
		  Eine Revision ist eine neue Version eines Artefaktes mit dem Zweck eine ältere abzulösen.
	\item \textbf{Baseline} \\
		  Eine Baseline ist ein Satz von Revisionen, d.h. ein Snapshot der Konfiguration.
	\item \textbf{Release} \\
		  Ein Release ist eine getestete und freigegebene Baseline.
\end{itemize}

\subsection{Sie können für ein kleines Entwicklungsprojekt Rahmen- und Sprint-Planung gemäss SoDa machen}

Nach klar!

\subsection{Sie können Sprintbacklogs für ein kleines Team formulieren, schätzen und geeignete Abnahmekriterien festlegen}

WI zügs... (Scrum Do sucks!) Anm. des Autors: Warum hat Fabian ScrumDo nicht gerne? Hat es ihm was getan?

\subsection{Sie können ein Controlling auf Projekt- und Sprintebene für kleine Entwicklungsprojekte gemäss SoDa führen}

Ist Chefsache!

\subsection{Sie wissen welche Informationen aus dem Entwicklungsprozess gemäss SoDa wann, wie und wo dokumentiert werden sollen}

Folgende Dokumente braucht es in einem SoDa-Projekt:

\begin{itemize}
	\item \textbf{Projektplan} \\
		  Erstellung in der Initialisierungsphase, Aktualisierung bei jedem Meilenstein
	\item \textbf{Grobkonzept} \\ 
		  Erstellung in der Initialisierungsphase
	\item \textbf{ProductBacklog} \\
		  Erstellung in der Initialisierungsphase, laufende Pflege
	\item \textbf{SprintBacklog} \\
		  Erstellung zu Beginn jedes Sprints => Teil des Sprintplans
	\item \textbf{Sprintplan} \\
		  Erstellung zu Beginn jedes Sprints => Ablage im Anhang Projektplan
	\item \textbf{SysSpec} \\
		  Erstellung in der Initialisierungsphase, Ergänzung pro Inkrement
	\item \textbf{Testplan} \\
		  Erstellung im Projektplan, Ergänzung pro Inkrement
	\item \textbf{Testprotokoll} \\
		  Erstellung pro Inkrement
\end{itemize}