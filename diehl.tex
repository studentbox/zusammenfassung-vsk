\section{Nebenläufigkeit, Verteilung, Synchronisation und Kommunikation (R. Diehl)}
\label{sec:diehl}

\subsection{Sie kennen den Lebenszyklus für Java Threads und deren Zustände}

\mygraphics{0.7\textwidth}{fig/thread_lebenszyklus.png}{Thread Lebenszyklus}{thread_lebenszyklus}

\subsection{Sie wissen wie ein Java Thread implementiert werden kann}

\begin{itemize}
	\item Von Klasse \texttt{Thread} ableiten
	\item \texttt{Runnable}-Interface implementieren
	\item Runnable-Objekt muss einem Thread im Konstruktor übergeben werden
	\item mit der Methode \texttt{start()} wird der Thread gestartet
\end{itemize}

\subsection{Sie kennen die drei Arten, wie ein Java Thread beendet werden kann}

\begin{enumerate}
	\item erzwungen (forceful cancellation) heisst sofortiger Abbruch
	\item verzögert (deferred cancellation) heisst Abbruch beim nächsten Abbruchpunkt
	\item kooperativ (cooperative cancellation) heisst Thread wird nur gebeten abzubrechen und muss sich dann selbst beenden, z.B. durch \texttt{return} in \texttt{run}-Methode
\end{enumerate}

\subsection{Sie können einen Thread aus einem andern Softwareteil abbrechen}

Ruft man die Methode \texttt{interrupt()} an einem Thread auf, wird ein Interrupt-Flag gesetzt. Innerhalb des Threads kann dieses Flag mit der Methode \texttt{Thread.currentThread.isInterrupted()} abgefragt werden und der Thread entsprechend beendet werden (z.B. mit \texttt{return}). Es gibt auch noch die statische Methode \texttt{Thread.interrupted()}, jedoch löscht diese Methode das Interrupt-Flag.

Eine \texttt{InterruptedException} wird von den Methoden \texttt{sleep()}, \texttt{join()} und \texttt{wait()} geworfen, wenn sie mit \texttt{interrupt()} unterbrochen wurden. Eine \texttt{InterruptedException} setzt das Interrupt-Flag zurück! Deshalb sollte im \texttt{catch}-Teil das Interrupt-Flag nochmals gesetzt werden.

\subsection{Sie kennen das Monitor Konzept}

Beginnen wir mit einem anschaulichen Alltagsbeispiel. Gehen wir aufs Klo, schliessen wir die Tür hinter uns. Möchte jemand anderes auf die Toilette, muss er warten. Vielleicht kommen noch mehrere dazu, die müssen dann auch warten, und eine Warteschlange bildet sich. Dass die Toilette besetzt ist, signalisiert die abgeschlossene Tür. Jeder Wartende muss so lange vor dem Klo ausharren, bis das Schloss geöffnet wird, selbst wenn der auf der Toilette Sitzende nach einer langen Nacht einnicken sollte.

Wie übertragen wir das auf Java? Soll ein Block nur von einem Thread gleichzeitig ausführbar, ist ein Monitor nötig. Ein Monitor wird mithilfe eines Locks (zu Deutsch »Schloss«) realisiert, welches ein Thread öffnet oder schliesst. Tritt ein Thread in den kritischen Abschnitt ein, wird der Bereich wie eine Tür abgeschlossen (engl. lock). Erst wenn der Abschnitt durchlaufen wurde, wird die Tür wieder aufgeschlossen (engl. unlock), und ein anderer Thread kann den Abschnitt betreten.

\subsection{Sie wissen was Reentrant und Nested Monitore sind und können diese vermeiden}

\subsubsection{Reentrant Monitor}

Betritt ein Thread eine synchronisierte Methode/Abschnitt, dessen Lock er schon besitzt, kann er sofort eintreten ohne den lock-Pool zu durchlaufen. Ohne diese Möglichkeit würden Rekursionen nicht funktionieren, da sonst ein Deadlock auftreten würde, da der Thread sozusagen wartet, bis er selbst den Lock wieder freigibt. 
Reentrant Monitore reduzieren aber auch die Parallelität, weil kritische Abschnitte künstlich vergrössert werden.

\subsubsection{Nested Monitor}

Bei einem Nested Monitor werden zwei unterschiedliche Monitore ineinander verschachtelt.

\begin{lstlisting}[caption={Nested Monitor},label=lst:nested_monitor]
public void lock() throws InterruptedException {
	synchronized(this){
		synchronized(this.monitorObject){
			this.monitorObject.wait();
		}
	}
}	  
\end{lstlisting}

Im obigen Beispiel wird zuerst das \texttt{this}-Objekt synchronisiert und danach das \texttt{monitorObject}. Es besteht nun Deadlock-Gefahr. Thread 1 hält den Lock für \texttt{this} und \texttt{monitorObject}. Thread 1 ruft \texttt{wait()} auf und gibt den Lock von \texttt{monitorObject} frei, jedoch nicht den Lock von \texttt{this}. Wenn jetzt Thread 2 den Lock von \texttt{this} benötigt um Thread 1 aufzuwecken kommt es zu einem Deadlock.

\subsection{Sie können ''Warten auf Bedingungen'' in Java umsetzten}

Ruft ein Thread an einem anderen Thread \texttt{wait()} auf wird dieser Schlafen gelegt (und gibt den Lock frei). Mit dem Aufruf von \texttt{notify()} wird ein Thread aufgeweckt und mit dem Aufruf von \texttt{notifyAll()} werden alle Threads aus dem Pool aufgeweckt. \texttt{notifyAll()} kann zu Performance-Problemen führen. 

Die Bedingung für \texttt{wait()} sollte immer in einer \texttt{while}-Schleife geprüft werden. Diese Massnahme verhindert dass der Thread aus Versehen aufgeweckt wird obwohl die Bedingung noch nicht eingetreten ist. 

\texttt{wait()} muss immer vor \texttt{notify()} aufgerufen werden, weil das Signal nicht gespeichert wird. Wird \texttt{notify()} aufgerufen und kein Thread wartet, wird auch keiner aufgeweckt.

\subsection{Sie kennen ausgewählte Synchronisationsmechanismen aus der java.util.concurrent Bibliothek}

\subsubsection{java.util.concurrent.Semaphore}

Dem Konstruktor von \texttt{Semaphore(permits)} muss mitgeteilt werden, wie viel von einer Ressource zur Verfügung steht (z.B. bei 5 Druckern stehen 5 Ressourcen bereit). Mit \texttt{aquire()} wird ein Zugang angefordert. Mit \texttt{release()} wird ein Zugang freigegeben.

\subsubsection{java.util.concurrent.BlockingQueue}

Dieses Interface bietet eine Queue (Puffer) an. Wenn die Queue voll ist wartet der Produzent (\texttt{put(element)}). Wenn die Queue leer ist wartet der Konsument (\texttt{take()}).

\subsubsection{java.util.concurrent.ThreadPoolExecutor}

Objekte welche das \texttt{Runnable}-Interface implementiert haben, können der Methode \texttt{execute(run)} übergeben werden. Der ThreadPoolExecutor verteilt die Run-Objekte auf mehrere Worker-Threads. 

\subsection{Sie wissen was Persistenz und Synchronität in der Kommunikation bedeutet}

\subsubsection{Persistenz}

Bei einer persistenten Kommunikation wird eine Nachricht so lange gespeichert bis der Empfänger bereit ist (z.B. E-Mail). Das Gegenteil ist transiente Kommunikation bei der die Nachricht nur verarbeitet wird solange der Empfänger ausgeführt wird (z.B. Router). 

\subsubsection{Synchronität}

Bei einer synchronen Kommunikation wird der Sender blockiert bis der Empfänger die Nachricht empfangen hat (z.B. Telefon). Synchrone Kommunikations ist bei zeitkritischen Anwendungen von Vorteil. Bei einer asynchronen Kommunikation wird der Sender fortgesetzt unmittelbar nach dem er seine Nachricht zur Übertragung weiter gegeben hat (z.B. Diskussionsforum). Der Sender kann sich bei der asynchronen Kommunikation um andere Aufgaben kümmern, weil er nicht blockiert ist.

\subsection{Sie wissen was eine Nachricht ist und kennen die Prinzipien der Nachrichtenverarbeitung}

Eine Nachricht enthält eine Anzahl Elemente von bestimmten Datentypen (meist eine ID und mehrere Argumente) und wird über einen Kommunikationskanal gesendet. Mithilfe von Nachrichten wird die eigentliche Applikation von den Kommunikationsdetails getrennt.

\subsection{Sie können Nachrichten für fixe oder adaptive Protokolle mit Hilfe des Basic- oder Adaptive-Message Typs nachvollziehen}

\subsubsection{Fixe Protokolle}

Bei fixen Kommunikationsprotokollen sind die Nachrichtenarten und ihre Argumente vor der Kommunikation bekannt. Die Nachrichten ändern sich auch nicht mehr während der Kommunikation. Die \texttt{BasicMessage} ist das Protokoll der Kommunikation und der \texttt{BasicMessageHandler} erstellt die entsprechenden Nachrichtentypen.

\subsubsection{Adaptive Protokolle}

Bei adaptiven Kommunikationsprotokollen ändert sich das Kommunikationsprotokoll während der Kommunikation. Der \texttt{MessageHandler} bewältigt die Änderungen des Nachrichten Protokolls.

\subsection{Sie kennen die Entwurfsmuster 'Fabrikmethode' und 'Prototyp'}

siehe Lernziel \ref{sec:patterns}

\subsection{Sie kennen zwei verschiedene Algorithmen zur Synchronisation von physischen Uhren}

\subsubsection{Algorithmus von Cristian}

Ein Zeitserver sendet am Anfang jeder UTC-Sekunde einen kurzen Impuls. Die Clients werden mit dem Server synchronisiert. Periodisch fragen die Clients den Zeitserver nach der Zeit. Der Server versucht so schnell wie möglich zu antworten. Die Dauer vom Senden zum Empfangen wird gemessen und halbiert, um die Laufzeit der Anfrage auszugleichen. Liegt die Zeit des Servers hinter der Zeit des Clients, wird die Zeit des Clients verlangsamt (Uhr darf nicht rückwärts laufen!).

\subsubsection{Berkeley-Algorithmus}

Der Zeit-Server fragt periodisch alle Clients nach ihrer Zeit. Basierend auf den Antworten berechnet er die Durchschnittszeit und weist alle Clients an diese zu übernehmen.

\subsection{Sie wissen was eine logische Uhr ist}

\begin{itemize}
	\item Es reicht aus wenn sich alle Maschinen über die gleiche Zeit einig sind
	\item Die Maschinen müssen nicht unbedingt nach UTC synchronisiert werden
	\item Wird vor allem eingesetzt wenn Abhängigkeiten und Verlässlichkeit wichtig sind
\end{itemize}

\subsection{Sie kennen die Happened-Before-Relation}

\begin{itemize}
	\item $a \rightarrow b$ bedeutet ''a passiert vor b'' (alle Prozesse sind sich einig das Ereignis a vor Ereignis b stattgefunden hat)
	\item Ereignisse sind kausal unabhängig wenn $a \rightarrow b$ und $b \rightarrow a$ nicht gilt ($a \parallel b$)
	\item Transitive Relation: wenn $a \rightarrow b$ und $b \rightarrow c$ gilt auch $a \rightarrow c$
\end{itemize}

\subsection{Sie kennen den Algorithmus des Lamport-Zeitstempels zur Synchronisation von logischen Uhren}

\begin{itemize}
	\item Jede Maschine hat eine eigene Uhr mit konstanten aber unterschiedlichen Geschwindigkeiten
	\item Jedem Ereignis wird ein Zeitwert zugeordnet
	\item Dieser Zeitwert sendet ein Prozess 1 an Prozess 2
	\item Wenn der Timer von Prozess 2 kleiner ist als der von Prozess 1, wird der Timer von Prozess 2 auf den Timer von Prozess 1 gestellt und 1 dazuaddiert
	\item Es können also nie zwei Ereignisse zur gleichen Zeit auftreten
	\item Mit einem Lamport-Zeitstempel lässt sich nicht bestimmen ob zwei Ereignisse kausal voneinander abhängig sind
\end{itemize}

\mygraphics{0.4\textwidth}{fig/lamport.png}{Lamport-Zeitstempel mit drei Prozessen}{lamport}

\subsection{Sie kennen den Algorithmus des Vektor-Zeitstempels zur Synchronisation von logischen Uhren}

\begin{itemize}
	\item Jeder Prozess besitzt einen Vektor, der für jeden Prozess im System die Anzahl Ereignisse festhält
	\item Der Vektor wird mit jeder Nachricht mitgegeben
	\item Tritt bei einem Prozess ein Ereignis auf inkrementiert er seinen Zähler im Vektor
	\item Der Vektor-Zeitstempel informiert den Prozess wie viele Ereignisse bei sich und bei anderen Prozessen vorausgegangen sind
	\item Dadurch lässt sich bestimmen ob zwei Ereignisse möglicherweise kausal voneinander abhängig sind
\end{itemize}

\mygraphics{0.7\textwidth}{fig/vektor.png}{Vektor-Zeitstempel}{vektor}

\subsection{Sie können die Algorithmen zur Synchronisation von logischen Uhren in einem Programm nachvollziehen}

\begin{itemize}
	\item Prozess 1 versendet eine Nachricht an Prozess 2 und schickt seinen Zeitstempel mit
	\item Liegt der Timer von Prozess 2 hinter dem Timer von Prozess 1 übernimmt Prozess 2 den Timer-Wert von Prozess 1 und addiert 1 dazu
	\item Beispiel-Anwendung:
	
	\href{http://geekum.wordpress.com/2012/02/22/code-for-lamport-timestamps-algorithm-in-java/}{http://geekum.wordpress.com/2012/02/22/code-for-lamport-timestamps-algorithm-in-java/}
\end{itemize}
