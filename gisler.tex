\section{Entwicklungsprozess (R. Gisler)}
\label{sec:gisler}

\subsection{Sie kennen die grundlegenden Aufgaben eines Versionskontrollsystems (VCS) und können damit arbeiten}
\subsection{Sie wissen was ein automatisierter Buildprozess ist und kennen seine Vorteile}

Alle Tätigkeiten (Kompilieren, Testen, Packen usw.) des Buildprozesses werden automatisch ausgeführt. Durch die Automatisierung ergeben sich folgende Vorteile:

\begin{itemize}
	\item weniger Fehler treten auf
	\item Ergebnisse sind reproduzierbar
	\item unabhängig von IDE
	\item lange Builds können z.B. über Nacht laufen
\end{itemize}

\subsection{Sie kennen die Motivation von Continuous Integration (CI) und sind in der Lage Projekte entsprechend zu entwickeln}

\begin{itemize}
	\item immer eine lauffähige Software haben
	\item Fehler sobald als möglich entdecken (durch Tests)
	\item parallel im Team entwickeln ohne Überblick zu verlieren
\end{itemize}

\subsection{Sie wissen was ein Buildserver ist und welche Bedeutung er für die Entwicklungstätigkeit hat}

Ein Buildserver ist eine Software die Quellcode automatisch buildet und die Resultate allen Entwickler zur Verfügung stellt. Der Build wird z.B. durch eine Änderung im Repo ausgelöst. Durch einen Buildserver ergeben sich folgende Vorteile:

\begin{itemize}
	\item Entlastung von Entwicklern von repetitiven Aufgaben
	\item Häufigere Verifikation (Buildprozess, Tests, Deploying etc.)
	\item Statistische Information über Entwicklungsprozess
	\item Offensive (automatische) Information über den Zustand der Projekte
\end{itemize}

\subsection{Sie kennen die verschiedenen Testarten (mindestens Unit~ und Integration~) und sind in der Lage gute Unittests zu schreiben}

\subsubsection{Unittest}

Schneller und automatisierter Test von kleinen Einheiten (z.B. Klasse). Es wird nur die Klasse und ohne ihre Abhängigkeiten getestet! Unittests werden während der Entwicklung meist mit Frameworks (z.B. JUnit, Mockito) erstellt und sind in die IDE integriert.

\subsubsection{Integrationstest}

Integrationstests werden durchgeführt wenn eine neue Komponente in eine bestehende Software-Landschaft integriert wird. Bei Integrationstests wird eher der technische Aspekt überprüft (z.B. korrekte Installation).

\subsubsection{Systemtest}

Bei den Systemtest wird das ganze System mittels einer Spezifikation getestet. Hier wird eher der Benutzeraspekt getestet (z.B. Usability).

Integrations- und Systemtests müssen explizit in einer Phase geplant, organisiert und durchgeführt werden (klassisches Testen)


\subsection{Sie beherrschen die Entwicklung nach dem Test First Prinzip}

Die meisten Entwickler testen um Fehler zu finden. Das Test First Prinzip findet aber Fehler bevor sie gemacht werden. 
Dazu muss man einfach die Tests vor der Implementierung schreiben (es funktioniert wirklich!). 
Weil man zuerst die Tests schreibt denkt man eher an Ausnahmefälle und kann diese später im Code besser berücksichtigen.

\subsection{Sie nutzen Werkzeuge zur Messung der Codeabdeckung aktiv zu Verbesserung Ihres Codes und der Testfälle}

Codeabdeckungswerkzeuge:
\begin{itemize}
	\item JaCoCo / EclEmma
	\item JCobertura
	\item Clover
\end{itemize}
Beim Fehlschlagen eines Testfalls kann so einfach nachgeforscht werden, welche Codezeilen ausgeführt wurden, und welche nicht, was ein Hinweis auf einen Fehler sein könnte.

\subsection{Sie haben einen Überblick über die Designpatterns nach GoF und kennen die verschiedenen Kategorien}


Entwurfsmuster können nach folgenden Gruppen eingeteilt werden:
\begin{itemize}
  \item \textbf{Erzeugungsmuster} \\
  Abstrahieren das Erstellen von Objekten.
  \item \textbf{Strukturmuster} \\
  Fassen Objekte zu grösseren Strukturen zusammen oder verbinden sie.
  \item \textbf{Verhaltensmuster} \\
  Beschreiben die Interaktion zwischen Objekten, mit Kontrollflüssen etc. Können die Zuständigkeit / Kontrolle delegieren.
  
\end{itemize}

\subsection{Sie können die Patterns Singleton, Strategy, Fassade, Adapter, Fabrikmethode, Prototyp und Observer erklären und sinnvoll anwenden}
\label{sec:patterns}

\subsubsection{Singleton (Erzeugungsmuster)}
Stellt sicher, dass von einer Klasse immer nur genau ein Objekt vorhanden ist, mittels einem privaten Konstruktor und einer \verb|getInstance()|-Methode. Sparsam einsetzen, sonst enstehen sternförmige Abhängigkeiten zur Singletonklasse.
    Siehe Abbildung \ref{fig:Singleton}.
    \mygraphics{0.5\textwidth}{fig/singleton.png}{Singleton}{Singleton}

\subsubsection{Strategy (Verhaltensmuster)}\label{sec:strategypattern}
 Die Lösung für \verb|switch() case:| Statements oder wenn unterschiedliches Verhalten erriecht werden soll. 
      \mygraphics{0.5\textwidth}{fig/strategy.png}{Strategy}{strategy}
      Der Client muss allerdings alle Implementationen kennen und initialisiert den Kontext mit einer konkreten Implementation. Der Client ruft daraufhin einfach die \texttt{sortiere()}-Methode des Kontexts auf, um die Sortierung zu starten. Wenn ein anderer Algorithmus verwendet werden soll, so wird dem Kontext einfach eine andere Instanz des Algorithmus übergeben.

\subsubsection{Fassade (Strukturmuster)}
Stellt eine Zusammenfassung für mehrere Schnittstellen eines Subsystems dar. Die Facade weiss dann, welche Subklassen für was genau zuständig sind und delegiert die Anfragen entsprechend. \\
    \textbf{Vorteile:}
    \begin{itemize}
        \item Einfachere Verwendung des Subsystems
        \item Separation of Concerns
        \item Unabhängige Entwicklung
        \item Austauschbarkeit der Subsysteme da gerine Abhängigkeit untereinander
        \item Unterklassen haben keine Kopplung
    \end{itemize}
    \textbf{Nachteile:} \\
    \begin{itemize}
        \item Delegation ist notwendig
        \item Gefahr eines reinen Durchlauferhitzers
    \end{itemize}
    Siehe Abbildung \ref{fig:Facade}.
    \mygraphics{0.5\textwidth}{fig/facade.png}{Facade}{Facade}

\subsubsection{Adapter (Strukturmuster)}
\mygraphics{0.5\textwidth}{fig/adapter.png}{Adapter}{adapter}
Ausgangslage: Man hat eine existierende Klasse, die Schnittstelle ist aber unpassend. Nun macht man eine abstrakte Zielschnittstelle (vgl. Abb. \ref{fig:adapter} - Klasse \texttt{Ziel}) und leitet von dieser den Adapter ab, welche dann die eigentliche Operation aufruft bei der existierenden Klasse. Dem Client weiss dabei nur, dass es eine Zielschnittstelle gibt.

\subsubsection{Fabrikmethode (Erzeugungsmuster)}
Grunsätzlich abstrahiert die Fabrikmethode das Erstellen eines Objektes. Die Erstellung des Objektes passiert also via ein Interface, und die implementierenden Objekte des Interfaces entscheiden dann, wie genau und was für ein Objekt implementiert wird. \\
\textbf{Vorteile}
\begin{itemize}
  \item \textbf{Entkapselung Objekterzeugung} \\
  Kann hilfreich sein beim weiterentwickeln von Software-Bibliotheken. Zudem können Objekte von verschiedenen Klassen mit derselben Factory erstellt werden.
  \item \textbf{Bessere Semantik des Konstruktors} \\
  Der Name des Konstruktors ist häufig vorgegeben, die einer Methode natürlich nicht. Also können viel aussagekräftigere oder diversere Methoden definiert werden, um ein Objekt zu erzeugen. Als Beispiel \texttt{Color.createFromRGB(...)} und \texttt{Color.createFromHSV(...)}.
\end{itemize}
\textbf{Nachteile}
\begin{itemize}
  \item \textbf{Starke Kopplung Konkreter Erzeuger und konkretes Produkt} \\
  Für jedes neue konkrete Produkt braucht es also einen neuen konreten Erzeuger.
\end{itemize}


\mygraphics{0.5\textwidth}{fig/factory.png}{Fabrikmethode}{factory}

\subsubsection{Prototyp (Erzeugungsmuster)} 
Wenn ein gewisses Objekt bereits vorhanden ist und man mag dieses Objekt so sehr, dass man gerne noch ein zweites oder ein hundertes davon möchte, dann kann man zur Erzeugung das Prototyp-Muster anwenden. Es sieht in UML folgendermassen aus:

\mygraphics{0.5\textwidth}{fig/prototype.png}{Prototyp}{prototype}

Die zu klonenden Objekte leiten einfach von einer abstrakten Basisklasse ab, welche eine Methode \texttt{clone()} beinhaltet und liefert dann eine Kopie seiner selbst zurück. Wenn es dabei noch auf andere Objekte referenziert, kann es die auch nochmals kopieren und dann zurückliefern (sog.\emph{deep cloning}). Oder es belässt die Referenzen so wie sie sind (\emph{shallow cloning}).

\textbf{Vorteile}
\begin{itemize}
  \item Keine Subklasse je Objekt benötigt, welches man erstellen möchte (wie beim Factory-Pattern notwendig)
  \item Effizienter als das Objekt mit \texttt{new Object()} von Hand zu erstellen.
\end{itemize}

\textbf{Nachteile}
\begin{itemize}
  \item Die Erstellung einer Kopie eines Objektes kann aufwendig sein.
  \item Jede Unterklasse muss die Kopie-Operation implementieren. 
  \item Eventuelle Initialisierungen des kopierten Objekts müssen zusätzlich erfolgen.
\end{itemize}


\subsubsection{Observer (Verhaltensmuster)}
\mygraphics{0.5\textwidth}{fig/observer.png}{Observer Pattern}{observer}

Wir haben z.B. ein Programm, welches im Model die Daten ändert. Das GUI soll diese Daten dann anzeigen. Das GUI ist nun ein \emph{Konkreter Beobachter} und meldet sich als solcher beim \emph{Subjekt}, dem Model an.

Wenn es nun Änderungen im Model gegeben hat, meldet dies das Subjekt allen angemeldeten Beobachtern über z.B. die in der Abbildung \ref{fig:observer} gezeigte Methode \texttt{aktualisiere()}.


\subsection{Sie können die oben genannten Patterns in Java in entsprechend angepasster Form implementieren und sind sich der Unterschiede zu den originalen Designs im Klaren}

\subsubsection{Observer}
Java hat das als Event/Listener Modell umgesetzt. Aber in Java ist man nicht Beobachter einer ganzen Klasse, sondern eines bestimmten Events, z.B. dem Schliessen eines Fensters. Deswegen haben wir in Java Methoden wie \texttt{addWindowClosingListener(ActionListener listener)}. Die Eventquelle (z.B. die GUI Klasse) verwaltet dann diese Beobachte und bietet die Schnittstellen zur De/Registrierung an. 
\subsubsection{Unterschied zum originalen Design}
In Java hat man Interfaces, deswegen können die Abstrakten Klasse Beobachter und Subjekt gegen Interfaces augestauscht werden. Dies ermöglicht ein besseres Design.

Anmerk. des Autors: Was gibts da noch mehr? UML Klassen sind ja vorhanden, diese mit Code auszufüllen und da CopyPaste... bringts das?

\subsection{Sie wissen worauf man beim Einsatz von Patterns in der Praxis achten sollte}
Um die Entwurfsmuster richtig einsetzten zu können, muss man sie zuerst richtig gut kennen. Dann unbedingt sinnvoll und überlegt einsetzen. Denn ein Entwurfsmuster ist keine ultimative Generallösung und besser man setzt kein Entwurfsmuster ein, als das falsche.
\subsubsection{Auswahl des richtigen Entwurfsmusters}
\begin{itemize}
  \item \textbf{Kategorie wählen} \\
  Geht es um Erezugung, Struktur oder Verhalten? Da fallen schon einige weg, wenn dies klar ist.
  \item \textbf{Vor- und Nachteile abwägen} \\
  Aus den Vor- und Nachteilen das beste Entwurfsmuster auswählen, bei Zweifeln das, welches am meisten Flexibilität bietet.
  \item \textbf{Mit Beispielen arbeiten} \\
  Bietet das Entwurfsmuster tatsächlich die erhofften Vorteile?
  \item \textbf{Variieren!} \\
  Wie gesagt sind Entwurfsmuster keine Generallösungen, sondern ein Konzept, das je nach Umgebung auch \emph{wohlüberlegt} angepasst / optimiert wird. Vorsicht dabei, sonst wird evt. die Grundidee des Musters zerstört!
  \item \textbf{Kombinieren} \\
  Muster lassen sich natürlich untereinander kombinieren, um die effizienteste Lösung zu erreichen. 
\end{itemize}

\subsection{Sie kennen die wichtigen Aspekte des Deployments von Software}
\subsubsection{Organisation des Deployments}
Die grundsätzliche Planung eines (evt komplexeren als hier abgebildeten) Deployments kann natürlich mit UML geschehen.
\mygraphics{0.6\textwidth}{fig/deployment_uml.png}{UML Deployment Diagramm}{deployment_uml}

Der \emph{Node} stellt dabei einen Computer / Server / Hardware dar. Das \emph{Artifact} die konkret ausführbare Datei, welche auf dem \emph{Node} installiert (und ausgeführt) wird.

\mygraphics{0.3\textwidth}{fig/artefakt.png}{UML Diagramm Artefakt}{artefact_uml}

Das \emph{artifact} kann noch genauer spezifiert werden - welche Konfigurationsdateien / Abhängigkeiten sind da noch mit dabei? Um die Verbindung zum restlichen UML Diagramm herzustellen (Komponenten / Klassen und so) gibt es die \emph{manifest}-Beziehung (siehe Abbildung \ref{fig:manifest_uml}). Diese gibt an, welche Komponente das Artefakt beinhaltet.
\mygraphics{0.3\textwidth}{fig/manifest.png}{UML Diagramm Manifest}{manifest_uml}

\subsubsection{Verteilung}
Programmcode kommt selten allein. Für die Verteilung braucht es noch Installations- und Deinstallationsprograme, Default und Beispielkonfiguration, ein Installationsmedium (CD? Internet? USB-Stick?).
Die Verteilung kann auch vollständig automatisiert sein, z.B. via Java Web Start (anm. des Autors: oder einem Chef/Puppet Rezept).
\subsubsection{Installation}
Die Installation soll möglichst automatisert sein (also auch reproduzierbar). Dabei gibt es verschiedene Bedürfnisse - der Desktop User will ein hübsches GUI, der Server Admin eher ein automatiserbares Skript und der Entwickler / Tester braucht wieder etwas spezielles. Zudem ist je nach Umgebung eine andere Reihenfolge nötig (z.B. Windows muss zuerst eine Cygwin Umgebung installiert werden etc). 

Ach und die Deinstallation nicht vergessen, ja?
\subsubsection{Konfiguration}
So wenig wie möglich, so viel wie nötig! Sollte mit der Standardkonfiguration lauffähig sein, aber Anpassungen sollen dann schon gemacht werden können. z.B. Anbindung an LDAP oder Logging / Monitoring System, MySQL / PostgreSQL verwenden können, etc.

\subsubsection{Konfigurationsmanagement}
Welche Softwareversion läuft beim Kunden XY? Kann er reibungslos auf die neuste Version updaten? Wurde das getestet? Was hat er für Umsysteme? Auf was läuft denn die Software? Welche Versionen von unserer Software laufen mit welchen Versionen? ... siehe Abschnitt \ref{sec:konfigurationsmanagement}.
\subsubsection{Release Notes}
Release-Notes anzeigen bei einem Releasewechsel mit übersicht über geänderten Funktionsumfang / Vorbedingungen / Dateiformate. Dies muss meist manuell gemacht werden, da es eine qualitative Aussage über die Entwicklungsgeschichte ist. Vielleicht auch direkt Bezug auf Change-Requests / Bugs machen. Hilft den Anwendern abzuschätzen, ob sich ein Wechsel auf die neuste Version lohnt.

\subsubsection{Erwartungen}
Einfach nichts versprechen, was nicht gehalten werden kann! Kunde gibt sonst eine negative Rückmeldung! Bei der Kommunikation klar machen, was ist bei dem Release dabei und dies auch konsistent. Beim Deployment aber immer auch Suppport bieten - Murphys Law.

\subsection{Sie sind in der Lage ein Deployment adressatengerecht durchzuführen}
Ja klar.
--- Infos dazu?
\subsection{Sie beherrschen das dreistellige Versionskonzept (Semantic Versioning)}
\begin{itemize}
  \item \textbf{Major Release} (\textbf{X}.x.x)\\
  API oder fachliche Funktion hat sich verändert, Konfiguration nicht mehr kompatibel zu früheren Versionen - Anpassungen sind nötig.
  \item \textbf{Minor Release} (x.\textbf{X}.x)\\
  API oder fachliche Funktion hat sich erweitert, Konfiguration immer noch kompatibel zu früheren Funktionen - keine Anpassungen nötig.
  \item \textbf{Bugfix Release} (x.x.\textbf{X})\\
  Keine Veränderung am Funktionsumfang / API. Lediglich Verbesserungen von bestehendem. Direkter Einstatz möglich.
  
\end{itemize}

\subsection{Sie kennen die Motivation von Clean Code}
\begin{itemize}
  \item \textbf{Code-Qualität verbessern} \\
  Das ultimative Oberziel. Dies durch verschiedene Grade, welche auf einzelne, überschaubare Praktiken fokussieren, welche der Entwickler nach und nach durchläuft (und immer wieder repetiert).
  \item \textbf{Bewusstsein für Qualität schaffen} \\
  Eine verinnerlichte Grundhaltung für guten Code zu schaffen. Software-Entwicklung als echtes Handwerk vermitteln, mit einer Auswahl an Prinzipien, Praktiken und einem Wertesystem.
  \item \textbf{Die Erosion von Code verlangsamen} \\
   Verhindern kann auch Clean Code sie zwar nicht, denn nach ca. 10 Jahren ist sowieso jede Software hoffnungslos überfüllt. 
   \item \textbf{Verbreitung von Wissen} \\
   Erfahrungen soll man weitergeben - Clean Code Snacks Kurzpräsentationen.
\end{itemize}

\subsection{Sie können die Ergebnisse von Werkzeugen zur Messung der Code Qualität (z.B. Checkstyle, PMD etc.) sinnvoll interpretieren und nutzen}

\subsection{Sie kennen die Kriterien von Clean Code bezüglich der Implementation von guten Testfällen}
\subsubsection{Grundsätze von Unit Tests}
Ein sauberer = lesbarer Unit-Test ist es dann, wenn er klar, einfach und ausdrucksdicht ist.
ist. Die Vorgehensweise für einen UnitTest ist die folgende:
\begin{itemize}
  \item Erstellen der Testdaten (\textbf{A}rrange)
  \item Manipulieren der Testdaten (\textbf{A}ct)
  \item Verifaktion der Ergebnisse (\textbf{A}ssert)
\end{itemize}

\subsubsection{Definition Unit-Testing}
Ein Unit-Test ist ein automatisierter Test, welcher eine kleine Einheit eines System aufruft, und dann eine Behauptung macht was passieren soll und dies dann prüft. (Liebe Grüsse an Roger Diehl: Nein, Unit Tests brauchen keine Integrationsumgebung mit Netzwerk und so).
\subsubsection{Vorsätze für gute Testfälle}
\begin{itemize}
  \item \textbf{Nur ein \texttt{assert} pro Unittest} \\
  Dies erhöht die Anzahl der Testmethoden, d.h. sie werden einzeln kleiner und damit überschaubarer. 
  \item \textbf{Nur ein Konzept pro Test} \\
  Man hat dadurch viel selektivere Fehlermeldungen, wenn ein Test fehlschlägt und muss nicht lange suchen. Zudem ist er gezielter wiederholbar.
  \item \textbf{F.I.R.S.T} \\
  \begin{itemize}
      \item Fast - schnelle Tests um eine regelmässige Ausführung zu ermöglichen
      \item Independent - unabhängige Reihenfolge / Ausführung der Tests
      \item Repeatable - Tests müssen auf/in jeder Umgebung ausführbar sein.
      \item Self-Validating - Die Tests sollen einfach \texttt{success} / \texttt{failure zurückliefern.}
      \item Timely - Tests sollen vor dem eigentlichen Code geschrieben werden -> Test First!
  \end{itemize}
  \item \textbf{Möglichst viel Testen} \\
  Klar oder? Einfach alle Bedingungen / Varianten / Berechnugnen möglichst testen. Immer nach dem Aufwand abschätzen, aber sollte ein gutes Ziel sein.
  \item \textbf{Code Coverage Werkzeuge verwenden} \\
  Diese zeigen auf, welche Zeilen ausgeführt wurden und welche nicht - hilft bei der Fehlersuche.
  \item \textbf{Triviale Tests auch machen} \\
  Ja auch, getter und setter.
  \item \textbf{Tests nicht igonrieren / auskommentieren} \\
  Denn dies ist ein Warnsignal und man sollte doch kurz die Anforderungen checken.
  \item \textbf{Grenzbedingungen} \\
  Das sind doch die interessantesten Fälle - man denke an eine Addition - was passiert bei einem Überlauf?
  \item \textbf{Ein Fehler kommt selten allein} \\
  Funktion des Fehlers ausgiebig testen - evt ganzer Rattenschwanz an Fehlern - Murphys Law hasst uns alle.
  \item \textbf{Muster der Ergebnisse von Tests zur Diagnose nutzen} \\
  Klar oder?
\end{itemize}


\subsection{Sie kennen die Kriterien von Clean Code zur Implementation von guten Methoden}
\subsubsection{Funktionen klein halten}
Macht sie einzeln viel verständlicher. Häuft die Anzahl von Funktionen und Klassen, was wiederum die Testbarkeit wahnsinnig vereinfacht. Auf gar keinen Fall eine Funktion in Abschnitte unterteilen - lieber 2 oder mehr Funktionen draus machen.
In Zahlen ausgedrückt - am Besten so um 20 Zeilen - max. 100. Maximal 150 Zeichen breit. Und keine unübersichtlichen Verschachtelungen! Max 2 Ebenen.
\subsubsection{Pro Funktion nur eine Aufgabe}
Eine Funktion soll nur etwas machen.  Und ihr Name sollte möglichst präzise genau das vermitteln . Der Trick dabei: das Wörtchen \emph{to} verwenden.  
Nebeneffekte werden so ebenfalls vermieden - eine Passwortcheck-Funktion soll nicht noch eine Session-Initialisierung enthalten. 
Unter diesem Aspekt fallen auch Anweisungen / Abfragen - eine Funktion die ein Name einfügt, soll nicht noch einen \texttt{false}-Wert zurückgeben, wenn der Name bereits existiert. Besser trennen in eine Abfragefunktion und eine Zuweisungsfunktion. 
Wenn Fehler in einer Funktion aufreten - Exception werfen! Ansonsten wird der Rückgabewert nochmals ausgewertet werden und wird womöglich noch falsch interpretiert.
Eine Fehlerbehandlung ist übrigens auch eine Aufgabe, das heisst es gibt Funktionen die nur das machen.
\subsubsection{Nur eine Abstraktionsebene pro Funktion}
Also nicht grosse Funktionen wie \texttt{doMagic()} und eine Zeile später \texttt{System.out.println("Wert von Variable x: " + x)}.
\subsubsection{Keine switch - Anweisungen}
Die Funktion wird zu gross und verletzt die "Nur eine Aufgabe" Regel. Besser ein Switch durch polymorphe Konstrukte ersetzen. Siehe Abschnitt \ref{sec:strategypattern} - Strategy Pattern).
\subsubsection{Anzahl der Funktionsargumente minimieren}
Absolut maximal 3. Bereits ab 2 eine Gefahr der Verwechslung da. Dies erreicht man, indem man kleinere Klassen macht. Beispiel: \texttt{addiere(num1,num2)} wird zu \texttt{num1.addiere(num2)}. Ebenfalls keine Flags als Argumente machen (z.B. \texttt{writeFile(file, true)}). Lieber zwei Funktionen!
\subsubsection{Output-Argumente vermeiden}
Argumente sollen nur als Input für eine Funktionen dienen, an den übergebenen Objekten soll nichts verändert werden.
\subsubsection{Tote Funktionen löschen}
Wird eine Funktion nie aufgerufen - weg damit. Wenn sie wieder benötigt wird, hat man ja alles im VCS.