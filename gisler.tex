\section{Entwicklungsprozess (R. Gisler)}
\label{sec:gisler}

\subsection{Sie kennen die grundlegenden Aufgaben eines Versionskontrollsystems (VCS) und können damit arbeiten}
\subsection{Sie wissen was ein automatisierter Buildprozess ist und kennen seine Vorteile}

Alle Tätigkeiten (Kompilieren, Testen, Packen usw.) des Buildprozesses werden automatisch ausgeführt. Durch die Automatisierung ergeben sich folgende Vorteile:

\begin{itemize}
	\item weniger Fehler treten auf
	\item Ergebnisse sind reproduzierbar
	\item unabhängig von IDE
	\item lange Builds können z.B. über Nacht laufen
\end{itemize}

\subsection{Sie kennen die Motivation von Continuous Integration (CI) und sind in der Lage Projekte entsprechend zu entwickeln}

\begin{itemize}
	\item immer eine lauffähige Software haben
	\item Fehler sobald als möglich entdecken (durch Tests)
	\item parallel im Team entwickeln ohne Überblick zu verlieren
\end{itemize}

\subsection{Sie wissen was ein Buildserver ist und welche Bedeutung er für die Entwicklungstätigkeit hat}

Ein Buildserver ist eine Software die Quellcode automatisch buildet und die Resultate allen Entwickler zur Verfügung stellt. Der Build wird z.B. durch eine Änderung im Repo ausgelöst. Durch einen Buildserver ergeben sich folgende Vorteile:

\begin{itemize}
	\item Entlastung von Entwicklern von repetitiven Aufgaben
	\item Häufigere Verifikation (Buildprozess, Tests, Deploying etc.)
	\item Statistische Information über Entwicklungsprozess
	\item Offensive (automatische) Information über den Zustand der Projekte
\end{itemize}

\subsection{Sie kennen die verschiedenen Testarten (mindestens Unit~ und Integration~) und sind in der Lage gute Unittests zu schreiben}

\subsubsection{Unittest}

Schneller und automatisierter Test von kleinen Einheiten (z.B. Klasse). Es wird nur die Klasse und ohne ihre Abhängigkeiten getestet! Unittests werden während der Entwicklung meist mit Frameworks (z.B. JUnit, Mockito) erstellt und sind in die IDE integriert.

\subsubsection{Integrationstest}

Integrationstests werden durchgeführt wenn eine neue Komponente in eine bestehende Software-Landschaft integriert wird. Bei Integrationstests wird eher der technische Aspekt überprüft (z.B. korrekte Installation).

\subsubsection{Systemtest}

Bei den Systemtest wird das ganze System mittels einer Spezifikation getestet. Hier wird eher der Benutzeraspekt getestet (z.B. Usability).

Integrations- und Systemtests müssen explizit in einer Phase geplant, organisiert und durchgeführt werden (klassisches Testen)


\subsection{Sie beherrschen die Entwicklung nach dem Test First Prinzip}

Die meisten Entwickler testen um Fehler zu finden. Das Test First Prinzip findet aber Fehler bevor sie gemacht werden. 
Dazu muss man einfach die Tests vor der Implementierung schreiben (es funktioniert wirklich!). 
Weil man zuerst die Tests schreibt denkt man eher an Ausnahmefälle und kann diese später im Code besser berücksichtigen.

\subsection{Sie nutzen Werkzeuge zur Messung der Codeabdeckung aktiv zu Verbesserung Ihres Codes und der Testfälle}

Codeabdeckungswerkzeuge:
\begin{itemize}
	\item JaCoCo / EclEmma
	\item JCobertura
	\item Clover
\end{itemize}
Beim Fehlschlagen eines Testfalls kann so einfach nachgeforscht werden, welche Codezeilen ausgeführt wurden, und welche nicht, was ein Hinweis auf einen Fehler sein könnte.

\subsection{Sie haben einen Überblick über die Designpatterns nach GoF und kennen die verschiedenen Kategorien}
Entwurfsmuster können nach folgenden Gruppen eingeteilt werden:
\subsubsection{Erzeugungsmuster}
Abstrahiert das Erstellen von Objekten.
\begin{itemize}
    \item Abstract Factory
    \item Builder
    \item Factor Method
    \item Prototype
    \item Singleton
    \subitem Stellt sicher, dass von einer Klasse immer nur genau ein Objekt vorhanden ist, mittels einem privaten Konstruktor und einer \verb|getInstance()|-Methode. Nachteil: Erzeugt mühsame Abhängigkeiten. 
    Siehe Abbildung \ref{fig:Singleton}.
    \mygraphics{0.5\textwidth}{fig/singleton.png}{Singleton}{Singleton}
    
\end{itemize}

\subsubsection{Strukturmuster}
Fassen Objekte zu grösseren Strukturen zusammen oder verbinden sie.
\begin{itemize}
    \item Adapter / Wrapper
    \item Bridge, Handle/Body
    \item Decorator, Wrapper
    \item Facade
    \subitem
    Stellt eine Zusammenfassung für mehrere Schnittstellen eines Subsystems dar. Die Fassade weiss dann, welche Subklassen für was genau zuständig sind und somit wird es einfacher, das Subsystem zu verwenden. Allerdings birgt es die Gefahr, ein Durchlauferhitzer zu werden, weil keine neue Funktionalität in das System bringt.
    \item Flyweight
    \item Composite
    \item Porxy / Surrogate
\end{itemize}
\subsubsection{Verhaltensmuster}
Beschreiben die Interaktion zwischen Objekten, mit Kontrollflüssen etc. Können die Zuständigkeit / Kontrolle delegieren.
\begin{itemize}
    \item Befehl (Kommando, Command, Action, Transaction)
    \item Beobachter (Observer, Dependents, Publish/Subscribe, Listener)
    \item Besucher (Visitor)
    \item Interpreter (Interpreter)
    \item Iterator (Iterator, Cursor)
    \item Memento (Memento, Token)
    \item Plugin (Plugin)
    \item Schablonenmethode (Template Method)
    \item Strategie (Strategy, Policy)
    \subitem Die Lösung für \verb|switch() case:| Statements. 
    \item Vermittler (Mediator)
    \item Zustand (State, Objects for States)
    \item Zuständigkeitskette (Chain of Responsibility)
\end{itemize}

\subsection{Sie können die Patterns Singleton, Strategy, Fassade, Adapter, Fabrikmethode, Prototyp und Observer erklären und sinnvoll anwenden}
\label{sec:patterns}
\subsection{Sie können die oben genannten Patterns in Java in entsprechend angepasster Form implementieren und sind sich der Unterschiede zu den originalen Designs im Klaren}
\subsection{Sie wissen worauf man beim Einsatz von Patterns in der Praxis achten sollte}
\subsection{Sie kennen die wichtigen Aspekte des Deployments von Software}
\subsection{Sie sind in der Lage ein Deployment adressatengerecht durchzuführen}
\subsection{Sie beherrschen das dreistellige Versionskonzept (Semantic Versioning)}
\subsection{Sie kennen die Motivation von Clean Code}
\subsection{Sie können die Ergebnisse von Werkzeugen zur Messung der Code Qualität (z.B. Checkstyle, PMD etc.) sinnvoll interpretieren und nutzen}
\subsection{Sie kennen die Kriterien von Clean Code bezüglich der Implementation von guten Testfällen}
\subsection{Sie kennen die Kriterien von Clean Code zur Implementation von guten Methoden}